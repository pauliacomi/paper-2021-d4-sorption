% !TEX root = ../manuscript.tex

Biogas, regarded as a promising renewable energy source, still needs to be
upgraded. This calls for the removal of the most prominent contaminants, among
others the octamethylcyclotetrasiloxane (D4) molecule. Herein, high throughput
computational screening in tandem with synthesis and adsorption testing revealed
the hydrophobic Zr-MOF PCN-777 as an optimal D4 adsorbent with record
gravimetric (\SI{1.8}{\gram\per\gram}) and volumetric
(\SI{0.49}{\gram\per\centi\metre\cubed}) uptakes, alongside a reversible and
fast adsorption/desorption process, good cyclability and easy regeneration. This
MOF was demonstrated to encompass an ideal combination of mesoporous cages and
chemical functionality to enable an optimal packing of the siloxane molecules
and their efficient removal while maintaining the process highly reversible
thanks to moderately high host/guest interactions. This work highlights the
efficacy of an integrated workflow for accelerating adsorbent selection for a
desired application, spanning the entire pipeline from method validation to
computational screening, synthesis and adsorption testing towards the
identification of the optimal adsorbents.
